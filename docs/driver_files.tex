% Created 2015-01-04 Sun 11:27
\documentclass[11pt]{article}
\usepackage[utf8]{inputenc}
\usepackage[T1]{fontenc}
\usepackage{fixltx2e}
\usepackage{graphicx}
\usepackage{longtable}
\usepackage{float}
\usepackage{wrapfig}
\usepackage{rotating}
\usepackage[normalem]{ulem}
\usepackage{amsmath}
\usepackage{textcomp}
\usepackage{marvosym}
\usepackage{wasysym}
\usepackage{amssymb}
\usepackage{capt-of}
\usepackage{hyperref}
\tolerance=1000
\author{L. Larrabee Strow}
\date{\today}
\title{Driver Files}
\hypersetup{
 pdfkeywords={},
  pdfsubject={},
  pdfcreator={Emacs 24.4.1 (Org mode 8.3beta)}}
\begin{document}

\maketitle
\tableofcontents


\section{Introduction}
\label{sec-1}

stands for "kCompressed Atmospheric Radiative Transfer Algorithm." This
is an infrared, "monochromatic" radiative transfer algorithm written for
a one dimensional non-scattering Earth atmosphere. More documentation is
found in "kcarta.pdf". This file shows the user how to set up the driver
files found in the "Test" subdirectory.

\section{Reminder about kCARTA database range}
\label{sec-2}

As given out, the code was optimized for the 605 - 2830 spectral range
which is the range covered by AIRS, IASA, CRiS, and HIRS and AERI
instruments. However the code is flexible enough to allow optical depth
and radiance calculations in other spectral bands. Since the FWHM of
lines gets smaller (larger) as the wavenumbers get smaller (larger), the
resolution of the database must change. Each file in each spectral range
will contain 10000 points; so for example at the default 0.0025
resolution of the main IR default band (605-2830 ), the files each span
25 . We envisage the following :

\begin{verbatim}
kcartachunks = 00080 : 0002.5 : 00150;  prefix = '/j';
kcartachunks = 00140 : 0005.0 : 00310;  prefix = '/k';
kcartachunks = 00300 : 0010.0 : 00510;  prefix = '/p';
kcartachunks = 00500 : 0015.0 : 00605;  prefix = '/q';
kcartachunks = 00605 : 0025.0 : 02830;  prefix = '/r'; ** default **
kcartachunks = 02830 : 0025.0 : 03580;  prefix = '/s';
kcartachunks = 03550 : 0100.0 : 05650;  prefix = '/m';
kcartachunks = 05550 : 0150.0 : 08350;  prefix = '/n';
kcartachunks = 08250 : 0250.0 : 12250;  prefix = '/o';
kcartachunks = 12000 : 0500.0 : 25000;  prefix = '/v';
kcartachunks = 25000 : 1000.0 : 44000;  prefix = '/u';
\end{verbatim}

\section{Test directory}
\label{sec-3}

Examples of two driverfiles, one which computes optical depths (based on
a list the user supplies), and the other which computes radiances (and
jacobians if asked). The user should carefully examine these files, as
they provide a working outline of how to use this package. This subdir
also includes two matlab files, containing radiances output using H2004
and H2008.

\subsection{Setting the paths}
\label{sec-3-1}

The user needs to supply paths to where the solar files, continuum
files, nlte files, klayers executables, optical depth database and
reference profiles are; this is controlled via

We will concentrate on the parameters that need to be set for kCARTA
runs spanning 605-2830 . The variables set in this file fall into three
categories:

\begin{enumerate}
\item Category A: Solar datafiles, water continuum files, 4um CO2 chifiles
\label{sec-3-1-0-1}

\begin{center}
\begin{tabular}{ll}
soldir & path to solar files\\
\hline
nltedir & path and name of NLTE files\\
co2ChiFilePath & path to 4 um CO2 files\\
cdir & path to continuum files\\
cswt,cfwt & self and forn continuum weights\\
\end{tabular}
\end{center}

\item Category B: Klayers executables and scratch space
\label{sec-3-1-0-2}

\begin{center}
\begin{tabular}{ll}
klayers$\backslash$\_code.junkdir & path to scratch space for klayers input/output\\
klayers$\backslash$\_code.aeri & path to klayers executable for AERI\\
 & (uplook, finer layers near ground)\\
klayers$\backslash$\_code.airs & path to klayers executable for AIRS,IASI/CRiS\\
 & (downlook, default klayers layers)\\
\end{tabular}
\end{center}

\item Category C: kCARTA database paths
\label{sec-3-1-0-3}

Depending on whether the user is using f77 binary files or Matlab binary
files (as set \(via\) iMatlab$\backslash$\_vs$\backslash$\_f77), the user is required to set the
path to the reference profile (that was used to generate the database),
as well as paths to where the database actually is. In addition, the
user can set paths to different flavors of the database (eg H2000,
H2004, H2008), depending on iHITRAN.

If iMatlab$\backslash$\_vs$\backslash$\_f77 == +1 then the Matlab files are

\begin{center}
\begin{tabular}{ll}
kpath & path to compressed files\\
refp & path to reference profile\\
\end{tabular}
\end{center}

while if iMatlab$\backslash$\_vs$\backslash$\_f77 == -1 then the f77 files are

\begin{center}
\begin{tabular}{ll}
kdatadir & path to ieee-le f77 binary data\\
kpathh2o & subdir to h2o files\\
kpathhDo & subdir to hDo files\\
kpathco2 & subdir to co2 files\\
kpathetc & subdir to all other gases\\
refp & path to reference profile\\
\end{tabular}
\end{center}
\end{enumerate}

\subsection{Setting the control variables}
\label{sec-3-2}

Depending what the user wants to do, the user sets the following
parameters in a separate file : which HITRAN version to use, start/stop
wavenumbers for the calculations, whether or not to do Jacobians, what
output units for the Jacobians, what CKD version, and name of input rtp
file. We will concentrate on textcolorblueuser$\backslash$\_set$\backslash$\_input$\backslash$\_downlook.m,
the file used to generate radiances and/or jacobians; the file used to
generate optical depths is very similar.

As above, the parameters set in the file can be divided into a number of
categories

Category A : HITRAN controllers

\begin{center}
\begin{tabular}{ll}
iHITRAN & sets the kCompressed directory\\
 & choices are H2000,H2004,H2008\\
iMatlab$\backslash$\_vs$\backslash$\_f77 & is the database is Matlab or ieee-le\\
 & iMatlab$\backslash$\_vs$\backslash$\_f77 = +1 use Matlab version\\
 & iMatlab$\backslash$\_vs$\backslash$\_f77 = -1 use f77 version\\
\end{tabular}
\end{center}

Category B : Jacobian controllers

Temperature jacobians are always done. Warning : the water jacobian
includes the effects of the continuum.\\

\begin{center}
\begin{tabular}{lll}
iDoJac & controls the jacobians gasids & (-1 for none)\\
 &  & no jacs\\
 & [iGid1 iGid2 \ldots{} iGidN]; & do jacobians for gases in list\\
\end{tabular}
\end{center}

iJacobOutput controls the output jacobians units\\

\begin{center}
\begin{tabular}{ll}
iJacobOutput = -1 & dr/dT, dr/dq\\
iJacobOutput = 0 & dr/dT, dr/dq*q\\
iJacobOutput = +1 & dBT/dT, dBT/dq*q\\
\end{tabular}
\end{center}

Category C : General controllers

\begin{center}
\begin{tabular}{lll}
iAirs & drives klayers & \\
 & 0,-1,+1,2,3 & raw, AIRS, IASI, CRiS\\
fA,fB & start, stop wavenumbers & in\\
 & make sure they span only one spectral range & \\
CKD & water continuum version & 1,2,3,4,5\\
 & 1,2 are the MT CKD versions & \\
 & 3,4,5 are modified MT CKD versions & \\
\end{tabular}
\end{center}

Category D : Profile information

\begin{center}
\begin{tabular}{ll}
dirin & the directory where the file is in\\
fin & is the actual file\\
iProfRun & which of the rtp profiles to run\\
\end{tabular}
\end{center}

\subsection{Running the code!}
\label{sec-3-3}

Finally the user can commence the computation, calling one or the other
of the routines named below (which call relevant files from above).

\begin{verbatim}
dokcarta_downlook.m              compute RT
dokcarta_opticaldepths.m         compute optical depths
\end{verbatim}

All the user has to do is make sure the correct user$\backslash$ files are called,
at the top of these files.

\section{VariablePressure}
\label{sec-4}

This contains the main files a user should need This makes the code(s)
slower. The structure and content of the directories is the same as
before \(viz\)

\begin{verbatim}
drwxr-xr-x 2 sergio pi_strow    10 Mar 24 04:49 Test
drwxr-xr-x 6 sergio pi_strow     8 Mar 23 11:58 private
drwxr-xr-x 3 sergio pi_strow     4 Mar 23 10:36 JACUP_VarPress
drwxr-xr-x 3 sergio pi_strow     4 Mar 23 10:35 JACDOWN_VarPress
\end{verbatim}

\(Test\) has dokcarta$\backslash$\_downlook.m, dokcarta$\backslash$\_uplook.m (very similar to the
"downlook" case) and dokcarta$\backslash$\_opticaldepths.m.\\

\(JADOWN\_VarPress\) has jacobian routines for downlooking instruments\\

\(JACUP\_VarPress\) has jacobian routines for uplooking instruments\\
% Emacs 24.4.1 (Org mode 8.3beta)
\end{document}
%%% Local Variables:
%%% mode: latex
%%% TeX-master: t
%%% End:
