\documentclass[12pt]{article}

%%%\usepackage{lucbr,graphicx,fancyhdr,longtable}
%\usepackage{lucidabr,graphicx,fancyhdr,longtable}
\usepackage{graphicx,fancyhdr,longtable}

\newcommand{\kc}{\textsf{kCARTA}\xspace}
\newcommand{\kl}{\textsf{kLAYERS}\xspace}
\newcommand{\cm}{\hbox{cm}}

% make a single, doubly indented line 
% (mainly used for driver file examples)
\newcommand{\ttab}{\indent\indent}

\input ASL_defs

\setlength{\textheight}{7.5in}
\setlength{\topmargin}{0.25in}
\setlength{\oddsidemargin}{.375in}
\setlength{\evensidemargin}{.375in}
\setlength{\textwidth}{5.75in}

\newlength{\colwidth}
\setlength{\colwidth}{8cm}
\newlength{\colwidthshort}
\setlength{\colwidthshort}{6cm}

\definecolor{light}{gray}{0.75}

\pagestyle{fancy}

% \date{July 5, 1994} % if you want a hardcoded date

\lhead{\textbf{\textsf{DRAFT}}}
\chead{kCARTA}
\rhead{\textsf{Version 1.0}}
\lfoot{UMBC}
\cfoot{}
\rfoot{\thepage}

\newcommand{\HRule}{\rule{\linewidth}{1mm}}
\newcommand{\HRulethin}{\rule{\linewidth}{0.5mm}}

\begin{document}
\thispagestyle{empty}
\vspace{2.0in}

\noindent\HRule
\begin{center}
\Huge \textbf{\textsf{kCARTA}}: An Atmospheric Radiative Transfer Algorithm 
using Compressed Lookup Tables
\end{center}
\noindent\HRule

\vspace{0.75in}
\begin{center}
\begin{Large}
Sergio De Souza-Machado, L. Larrabee Strow,\\ Howard Motteler and Scott
Hannon
\end{Large}
\end{center}

\vspace{0.5in}
\begin{center}
Physics Department\\
University of Maryland Baltimore County\\Baltimore, MD 21250 USA\\
\end{center}

\vspace{0.5in}
\begin{center}
Copyright 2011 \\
University of Maryland Baltimore County \\
All Rights Reserved\\
v1.00  \today\\
\end{center}

\vfill

\noindent\HRulethin
\begin{flushleft}
\begin{tabbing}
Sergio~De~Souza-Machado: \=    sergio@umbc.edu \\
L.~Larrabee~Strow:   \>        strow@umbc.edu\\
\end{tabbing}
\end{flushleft}

%\begin{flushright}
%\includegraphics[width=1.0in]{umseal.eps}
%\end{flushright}

\newpage
\tableofcontents
%\listoftables
%\listoffigures
\newpage

\newpage
\section{Introduction}

\kc stands for ``kCompressed Atmospheric Radiative Transfer
Algorithm.''  This is an infrared, ``monochromatic'' radiative
transfer algorithm written for a one dimensional non-scattering Earth
atmosphere. More documentation is found in "kcarta.pdf". This file 
shows the user how to set up the driver files found in the "Test" subdirectory.

\section{Reminder about kCARTA database range}
As given out, the code was optimized for the 605 - 2830 \wn spectral range 
which is the range covered by AIRS, IASA, CRiS, and HIRS and AERI instruments.
However the code is flexible enough to allow optical depth and radiance 
calculations in other spectral bands. Since the FWHM of lines gets smaller 
(larger) as the wavenumbers get smaller (larger), the resolution of the 
database must change. Each file in each spectral range will contain 10000 
points; so for example at the default 0.0025 \wn resolution of the main IR 
default band (605-2830 \wn), the files each span 25 \wn. We envisage the 
following :

\begin{verbatim}
  kcartachunks = 00080 : 0002.5 : 00150;  prefix = '/j';
  kcartachunks = 00140 : 0005.0 : 00310;  prefix = '/k';
  kcartachunks = 00300 : 0010.0 : 00510;  prefix = '/p';
  kcartachunks = 00500 : 0015.0 : 00605;  prefix = '/q';
  kcartachunks = 00605 : 0025.0 : 02830;  prefix = '/r'; ** default **
  kcartachunks = 02830 : 0025.0 : 03580;  prefix = '/s';
  kcartachunks = 03550 : 0100.0 : 05650;  prefix = '/m';
  kcartachunks = 05550 : 0150.0 : 08350;  prefix = '/n';
  kcartachunks = 08250 : 0250.0 : 12250;  prefix = '/o';
  kcartachunks = 12000 : 0500.0 : 25000;  prefix = '/v';
  kcartachunks = 25000 : 1000.0 : 44000;  prefix = '/u';
\end{verbatim}

\section{Test directory}
Examples of two driverfiles, one which computes optical depths (based on a 
list the user supplies), and the other which computes radiances (and 
jacobians if asked). The user should carefully examine these files, as they
provide a working outline of how to use this package.
This subdir also includes two matlab files, containing radiances output
using H2004 and H2008.

\subsection{Setting the paths}
The user needs to supply paths to where the solar files, continuum files,
nlte files, klayers executables, optical depth database and reference profiles
are; this is controlled via \textcolor{blue}{$user\_set\_dirs.m$}

We will concentrate on the parameters that need to be set for kCARTA runs
spanning 605-2830 \wn. The variables set in this file fall into three
categories : \\

\vspace{0.2in}
\noindent Category A : solar datafiles, water continuum files, 4um CO2 chifiles\\
\vspace{0.2in}
\begin{tabular}{ll}
\hline
soldir &  path to solar files \\
\hline
nltedir &  path and name of NLTE files\\
co2ChiFilePath & path to 4 um CO2 files\\
\hline
cdir      & path to continuum files \\
cswt,cfwt & self and forn continuum weights \\
\end{tabular}

\vspace{0.2in}
\noindent Category B : klayers executables and scratch space \\
\vspace{0.2in}
\begin{tabular}{ll}
\hline
klayers\_code.junkdir & path to scratch space for klayers input/output\\
klayers\_code.aeri    & path to klayers executable for AERI \\
                     &      (uplook, finer layers near ground)\\
klayers\_code.airs    & path to klayers executable for AIRS,IASI/CRiS\\
                     &     (downlook, default klayers layers)\\
\hline
\end{tabular}

\vspace{0.2in}
\noindent Category C : kCARTA database paths \\
Depending on whether the user is using f77 binary files or Matlab binary files
(as set $via$ iMatlab\_vs\_f77), the user is required to set the path to the reference
profile (that was used to generate the database), as well as paths to where the
database actually is. In addition, the user can set paths to different flavors of
the database (eg H2000, H2004, H2008), depending on iHITRAN.

If iMatlab\_vs\_f77 == +1 then the Matlab files are \\
\vspace{0.2in}
\begin{tabular}{ll}
\hline
kpath & path to compressed files \\
refp  & path to reference profile\\
\hline
\end{tabular}

\newpage 
while if iMatlab\_vs\_f77 == -1 then the f77 files are\\ 
\vspace{0.2in}
\begin{tabular}{ll}
\hline
kdatadir & path to ieee-le f77 binary data \\
kpathh2o & subdir to h2o files\\
kpathhDo & subdir to hDo files\\
kpathco2 & subdir to co2 files\\
kpathetc & subdir to all other gases\\
refp  & path to reference profile\\
\hline
\end{tabular}

\subsection{Setting the control variables}
Depending what the user wants to do, the user sets the following parameters 
in a separate file : which HITRAN version to use, start/stop wavenumbers for
the calculations, whether or not to do Jacobians, what output units for the
Jacobians, what CKD version, and name of input rtp file. We will concentrate
on textcolor{blue}{user\_set\_input\_downlook.m}, the file used to generate 
radiances and/or jacobians; the file used to generate optical depths is very similar.

As above, the parameters set in the file can be divided into a number
of categories

\vspace{0.2in}
\noindent Category A : HITRAN controllers\\
\vspace{0.2in}
\begin{tabular}{ll}
\hline
iHITRAN & sets the kCompressed directory\\
        & choices are H2000,H2004,H2008\\
iMatlab\_vs\_f77 & is the database is Matlab or ieee-le \\\
  & iMatlab\_vs\_f77 = +1   use Matlab version \\
  & iMatlab\_vs\_f77 = -1   use f77 version\\
\end{tabular}

\vspace{0.2in}
\noindent Category B : Jacobian controllers\\
Temperature jacobians are always done. Warning : the water
jacobian includes the effects of the continuum.\\
\vspace{0.2in}
\begin{tabular}{lll}
\hline
iDoJac & controls the jacobians gasids & (-1 for none) \\
       &                               & no jacs \\
       & [iGid1 iGid2 ... iGidN];      &  do jacobians for gases in list\\
\hline
\end{tabular}

iJacobOutput controls the output jacobians units\\
\vspace{0.2in}
\begin{tabular}{ll}
\hline
   iJacobOutput = -1 &        dr/dT, dr/dq \\
   iJacobOutput =  0 &        dr/dT, dr/dq*q \\
   iJacobOutput = +1 &        dBT/dT, dBT/dq*q \\
\hline
\end{tabular}

\vspace{0.2in}
\noindent Category C : General controllers\\
\vspace{0.2in}
\begin{tabular}{lll}
\hline
  iAirs & drives klayers & \\
        & 0,-1,+1,2,3  &  raw, AIRS, IASI, CRiS\\
\hline
fA,fB & start, stop wavenumbers & in \wn \\
      & make sure they span only one spectral range & \\%
\hline
 CKD & water continuum version & 1,2,3,4,5 \\
     & 1,2 are the MT CKD versions & \\
     & 3,4,5 are modified MT CKD versions & \\
\end{tabular}

\vspace{0.2in}
\noindent Category D : Profile information\\
\vspace{0.2in}
\begin{tabular}{ll}
\hline
   dirin & the directory where the file is in\\
   fin   & is the actual file\\
  iProfRun & which of the rtp profiles to run\\
\end{tabular}

\subsection{Running the code!}
Finally the user can commence the computation, calling one or the other
of the routines named below (which call relevant files from above).
\begin{verbatim}
dokcarta_downlook.m              compute RT
dokcarta_opticaldepths.m         compute optical depths
\end{verbatim}
All the user has to do is make sure the correct user\_* files are called,
at the top of these files.

\section{VariablePressure \textcolor{red}{Different pressure layers}}
This contains the main files a user should need \textcolor{red}{for a 
pressure layering different than the AIRS 100 layers.} This makes the code(s)
slower. The structure and content of the directories is the same as before 
$viz$

\begin{verbatim}
drwxr-xr-x 2 sergio pi_strow    10 Mar 24 04:49 Test
drwxr-xr-x 6 sergio pi_strow     8 Mar 23 11:58 private
drwxr-xr-x 3 sergio pi_strow     4 Mar 23 10:36 JACUP_VarPress
drwxr-xr-x 3 sergio pi_strow     4 Mar 23 10:35 JACDOWN_VarPress
\end{verbatim}

\noindent $Test$ has dokcarta\_downlook.m, dokcarta\_uplook.m (very similar to 
the "downlook" case) and dokcarta\_opticaldepths.m.\\

\noindent $JADOWN\_VarPress$ has jacobian routines for downlooking 
instruments\\

\noindent $JACUP\_VarPress$ has jacobian routines for uplooking instruments\\

\end{document}
