\documentclass[12pt]{article}

%%%\usepackage{lucbr,graphicx,fancyhdr,longtable}
%\usepackage{lucidabr,graphicx,fancyhdr,longtable}
\usepackage{graphicx,fancyhdr,longtable}

\newcommand{\kc}{\textsf{kCARTA}\xspace}
\newcommand{\kl}{\textsf{kLAYERS}\xspace}
\newcommand{\cm}{\hbox{cm}}

% make a single, doubly indented line 
% (mainly used for driver file examples)
\newcommand{\ttab}{\indent\indent}

\input ASL_defs

\setlength{\textheight}{7.5in}
\setlength{\topmargin}{0.25in}
\setlength{\oddsidemargin}{.375in}
\setlength{\evensidemargin}{.375in}
\setlength{\textwidth}{5.75in}

\newlength{\colwidth}
\setlength{\colwidth}{8cm}
\newlength{\colwidthshort}
\setlength{\colwidthshort}{6cm}

\definecolor{light}{gray}{0.75}

\pagestyle{fancy}

% \date{July 5, 1994} % if you want a hardcoded date

\lhead{\textbf{\textsf{DRAFT}}}
\chead{kCARTA}
\rhead{\textsf{Version 1.0}}
\lfoot{UMBC}
\cfoot{}
\rfoot{\thepage}

\newcommand{\HRule}{\rule{\linewidth}{1mm}}
\newcommand{\HRulethin}{\rule{\linewidth}{0.5mm}}

\begin{document}
\thispagestyle{empty}
\vspace{2.0in}

\noindent\HRule
\begin{center}
\Huge \textbf{\textsf{kCARTA}}: An Atmospheric Radiative Transfer Algorithm 
using Compressed Lookup Tables
\end{center}
\noindent\HRule

\vspace{0.75in}
\begin{center}
\begin{Large}
Sergio De Souza-Machado, L. Larrabee Strow,\\ Howard Motteler and Scott
Hannon
\end{Large}
\end{center}

\vspace{0.5in}
\begin{center}
Physics Department\\
University of Maryland Baltimore County\\Baltimore, MD 21250 USA\\
\end{center}

\vspace{0.5in}
\begin{center}
Copyright 2011 \\
University of Maryland Baltimore County \\
All Rights Reserved\\
v1.00  \today\\
\end{center}

\vfill

\noindent\HRulethin
\begin{flushleft}
\begin{tabbing}
Sergio~De~Souza-Machado: \=    sergio@umbc.edu \\
L.~Larrabee~Strow:   \>        strow@umbc.edu\\
\end{tabbing}
\end{flushleft}

%\begin{flushright}
%\includegraphics[width=1.0in]{umseal.eps}
%\end{flushright}

\newpage
\tableofcontents
%\listoftables
%\listoffigures
\newpage

\newpage
\section{Introduction}

\kc stands for ``kCompressed Atmospheric Radiative Transfer
Algorithm.''  This is an infrared, ``monochromatic'' radiative
transfer algorithm written for a one dimensional non-scattering Earth
atmosphere. More documentation is found in "kcarta.pdf". This file 
shows the user how to convolve the output from the Matlab kCARTA runs.

\section{Reminder about kCARTA output}

As given out, the code was optimized for the 605 - 2830 \wn spectral range 
which is the range covered by AIRS, IASA, CRiS, and HIRS and AERI instruments.
The spectral convolutions we describe in this section are designed for this
range. In general, the output is of the form
\begin{verbatim}
  radsOut or jacsOut or odOut
\end{verbatim}

where for example the fields of the structure are
\begin{verbatim}
    freqAllChunks                  1x90000          freq      cm-1
     radAllChunks                  90000x1          radiances mW/cm2/sr/cm-1
 iaa_kcomprstats_AllChunks          2x73            Singular Vectors stats
\end{verbatim}

"jacOut" and "odOut" will have fields that are for example
\begin{verbatim}
    ejacAllChunks: [90000x1]       surface emissivity jacobians
    qjacAllChunks: [2x90000x96]    gas amount jacs, for each gas in iDoJac
    sjacAllChunks: [90000x1]       surface temp jacobians
    tjacAllChunks: [90000x96]      temperature jacobians
     wgtAllChunks: [90000x96]      weighting functions
\end{verbatim}

We have provided some general purpose convolvers in the CONVOLVE subdirectory.
The user is free to modify the routines, at his/her own risk. Some of the routines
are contained within this package; if the user does wish to use them, he/she
will need to get more routines from us.

\begin{verbatim}
aeri_convolution_results.m          AERI convolver (needs fixing)
airs_convolution_results.m          AIRS convolver
cris_convolution_results.m          CRIS convolver
iasi_convolution_results.m          IASI convolver

kcarta_fconvkc.m                    sets up the FFT convolver

generic_convolution_results.m       gaussian convolver
quickconvolve.m                     called by generic_convolve

convolveNplot.m                     Calls one of the convolvers

rad2bt.m
\end{verbatim}

In everything described below, we assume we are doing either a radiance
convolution; other convolutions can be done similarly, by pulling out 
the appropriate fields of eg a jac "temperature" convolution

\section{AIRS convolution}
[rconv, fconv] = sconv2(rads,freqs,clist,sfile);

Here \\
sfile = path to AIRS SRFs\\
clist = list of AIRS channels that you want results for\\
freqs = input freqs from, radsOut structure\\
rads  = input radiances from radsOut structure\\

\section{Interferometer convolution}
rconv, fconv = kcarta\_fconvkc(rads,freqs,ifp,atype,aparg); \\
rconv, fconv = s1fconvkc(rads, ifp, atype, aparg); \\      
rconv, fconv = s2fconvkc(rads, ifp, atype, aparg); \\      
rconv, fconv = s3fconvkc(rads, ifp, atype, aparg); \\      

Here \\
freqs = input freqs from, radsOut structure\\
rads  = input radiances from radsOut structure\\
ifp = interferometer type\\
atype = apodization\\
aparg = argument (strength) of apodization\\

In general the matlab file "ifp" contains the start and stop wavenumbers,
(fA,fB) that are expected for the convolutions. If freqs corresponds exactly
to these parameters, then you can directly call sXfconvkc. The different 
flavors X=1,2,3 stand for \\
X = 1 : fast, not very accurate\\
X = 2 : compromise between 1 and 3 (the goldilocks optimal)\\
X = 3 : slow, very accurate\\

However if freqs only partially spans (fA,fB), or overspans (fA,fB), then
kcarta\_fconvkc tries to zero fill required data, or cut out unnecessary 
monochromatic data, before calling s2fconkc.

The parameters (ifp, atype, aparg) for various instruments are : \\
IASI : iasi12992','gauss',6 \\
CRIS B1 : 'crisB1','hamming',6\\
CRIS B2 : 'crisB2','hamming',6\\
CRIS B3 : 'crisB3','hamming',6\\
Note : you don't really need to supply "aparg" ie you can just use \\
  rconv, fconv = s3fconvkc(rads, ifp, atype);

Note that the fconvkc routines need to be separately called N times, if the
interferometer parameters are specified separately for different bands.

\section{Generic gaussian convolver}
  [fconv,rconv] = quickconvolve(freqs,rads,rFWHM,rSp); 

Here the user can input a FWHM and a channel spacing, and a generic gaussian
SRF is applied for convolution : \\
freqs = input freqs from, radsOut structure\\
rads  = input radiances from radsOut structure\\
rFWHM = FWHM of SRF model\\
rSP = channel spacing\\
\end{document}
